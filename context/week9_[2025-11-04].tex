% !TeX root = ../main.tex

\section{Homework \#9 [2025-11-04]}

\begin{problem}
  Calculate the Landau parameters to leading order in $\lambda_{1,2}$ for a
  Fermi liquid with the contact interactions
  \begin{enumext}
    \item $V(\bm x - \bm x') = \lambda_1\delta^{(3)}(\bm x - \bm x')$.
    \item $V(\bm x - \bm x') = -\lambda_2\nabla^2\delta^{(3)}(\bm x - \bm x')$
          (so that $V(q) = \lambda_1 q^2$ in Fourier space).
    \item Taking the results of (a) and (b) literally, sketch the regions of
          the $\lambda_1$, $\lambda_2$ phase diagram where the Fermi surface
          becomes unstable.
  \end{enumext}
\end{problem}
\begin{solution}\leavevmode
  For a spin-independent two-body interaction $V(\bm q)$,
  the Landau interaction function in the forward direction
  \[
    f_{\bm p\sigma, \bm p'\sigma'}
  = \ab[V(0) - V(|\bm p - \bm p'|)] \delta_{\sigma\sigma'}.
  \]
  Decomposing into spin-symmetric $f^s$ and spin-antisymmetric $f^a$ parts
  \[
    f^s(\theta) = V(0) - \frac12 V(q), \quad
    f^a(\theta) = -\frac12 V(q),
  \]
  where $q = |\bm p - \bm p'| = 2k_F \sin(\theta/2)$,
  $\theta = \braket<\bm p, \bm p'>$. Expanding in Legendre polynomials
  \[
    f^s(\theta) = \sum_l f_l^s P_l(\cos\theta), \quad
    f^a(\theta) = \sum_l f_l^a P_l(\cos\theta).
  \]
  Hence, the Landau parameters are
  \[
    F_l^s = N(0) f_l^s, \quad F_l^a = N(0) f_l^a.
  \]
  where $N(0) = \frac{m k_F}{2\pi^2}$ is the density of states per spin at the
  Fermi level.
  \begin{enumext}
    \item In the case of
    $V(\bm x - \bm x') = \lambda_1 \delta^{(3)}(\bm x - \bm x')$.\\
    Do the Fourier transform
    \[
      V(\bm q) = V(0) = \lambda_1,
    \]
    then
    \[
      f_l^s(\theta) = f_0^s = \lambda_1 - \frac{\lambda_1}{2}
    = \frac{\lambda_1}{2}, \quad
      f_l^a(\theta) = f_0^a = -\frac{\lambda_1}{2}.
    \]
    Hence, the Landau parameters are
    \[
      F_0^s = \frac{N(0)\lambda_1}{2},\quad
      F_0^a = -\frac{N(0)\lambda_1}{2},\quad
      F_{l>0} = 0.
    \]
    \item In the case of
    $V(\bm x - \bm x') = -\lambda_2\nabla^2\delta^{(3)}(\bm x - \bm x')$.\\
    Do the Fourier transform
    \[
      V(\bm q) = \lambda_2 q^2, \qq{and} V(0) = 0,
    \]
    then
    \[
      f^s(\theta) = f^a(\theta) = -\frac{\lambda_2}{2} q^2
    \xlongequal{q^2 = 2k_F^2 (1 - \cos\theta)}
      -\lambda_2 k_F^2 (1 - \cos\theta).
    \]
    Expanding in Legendre polynomials
    \[
      f^s(\theta) = -\lambda_2 k_F^2 + \lambda_2 k_F^2 \cos\theta
    = -\lambda_2 k_F^2 + \lambda_2 k_F^2 P_1(\cos\theta),
    \]
    Then
    \[
      f_0^s = f_0^a = -\lambda_2 k_F^2, \quad
      f_1^s = f_1^a = \lambda_2 k_F^2, \quad f_{l \geq 2} = 0.
    \]
    Hence, the Landau parameters are
    \[
      F_0^s = F_0^a = -N(0)\lambda_2 k_F^2,\quad
      F_1^s = F_1^a = N(0)\lambda_2 k_F^2,\quad F_{l>1} = 0.
    \]
    \item When both interactions are present, the Landau parameters add
    linearly
    \[
      F_0^s = \frac{N(0)\lambda_1}{2} - N(0)\lambda_2 k_F^2, \quad
      F_0^a = -\frac{N(0)\lambda_1}{2} - N(0)\lambda_2 k_F^2, \quad
      F_1^s = F_1^a = N(0)\lambda_2 k_F^2.
    \]
    The Pomeranchuk stability conditions are
    \[
      1 + \frac{F_l^s}{2l + 1} > 0, \quad 1 + \frac{F_l^a}{2l + 1} > 0.
    \]
    For $l = 0$: $1 + F_0^s > 0$, $1 + F_0^a > 0$;
    For $l = 1$: $1 + \frac{F_1^s}{3} > 0$, $1 + \frac{F_1^a}{3} > 0$.
    Define two coefficients $A = \frac{N(0)}{2}$, $B = N(0) k_F^2$, then
    \[
      F_0^s = A\lambda_1 - B\lambda_2, \quad
      F_0^a = -A\lambda_1 - B\lambda_2, \quad
      F_1^s = F_1^a = B\lambda_2.
    \]
    Let $x = A\lambda_1$, $y = B\lambda_2$, the stability conditions become
    \begin{enumext}[columns = 3]
      \item $1 + x - y > 0$
      \item $1 - x - y > 0$
      \item $1 + \frac y3 > 0$.
    \end{enumext}
    \begin{paracol}{2}
    Do the linear programming, the stable region in the $(x, y)$-plane is a
    triangle on the right.
    In terms of $\lambda_1, \lambda_2$, the stable region is inside the triangle
    with vertices
    \[
      \ab(-\frac4A, -\frac3B), \quad \ab(\frac4A,-\frac3B), \quad
      \ab(0,\frac1B).
    \]
    \switchcolumn \centering
    \begin{tikzpicture}[scale = .4]
      \draw [->] (-5,0) -- (5,0) node [below] {$x$};
      \draw [->] (0,-4) -- (0,2) node [left] {$y$};
      \filldraw [pattern = bricks, pattern color = violet]
        (-4,-3) node [below] {$(-4,-3)$} --
        (4,-3)  node [below] {$(4,-3)$} -- (0,1)
       node [right] {$(0,1)$} -- cycle;
    \end{tikzpicture}
    \end{paracol}
  \end{enumext}
\end{solution}

\begin{problem}
  Test your understanding of Landau's mass renormalization formula by
  generalizing it to include the effect of a magnetization. Suppose we introduce
  a second vector potential into \eqref{6.86}
  \begin{equation}
    A(\theta) \underset{|\bm k|\to0}\sim \int_{k_F-k\cos\theta}^{k_F} \d q
    \frac{2\iu\pi a^2}{\omega - v_F(|\bm k + \bm q| - q)}
  = \frac{2\iu\pi a^2k\cos\theta}{\omega - v_Fk\cos\theta}.
    \tag{6.86} \label{6.86}
  \end{equation}
  that couples to the spin current, writing
  \[
    \mathcal H[\mathbf A_N, \mathbf W]
  = \sum_\sigma \int \d^3x \frac1{2m} \psi_\sigma^\dagger(x)
    [(-\iu\hbar\nabla - \mathbf A_N - \sigma\mathbf W)^2] \psi_\sigma(x)
  + \hat V.
  \]
  Whereas $\mathbf A_N$ couples to the current of particles, $\mathbf W$
  couples to the ($z$ component of the) spin current.
  Assume that $V$ conserves spin current.
  \begin{enumext}
    \item By comparing the bare shift of the energies
    \[
      \delta\epsilon_{\bm p\sigma}^{(0)}
    = -\frac{\bm p}{m} \cdot (\mathbf A_N + \sigma \mathbf W),
    \]
    with the shift that results from interaction feedback,
    \[
      \delta\epsilon_{\bm p\sigma}
    = -\frac{\bm p}{m^*} \cdot \mathbf A_N
    - \sigma \frac{\bm p}{m_s^*} \cdot \mathbf W,
    \]
    show that there are two different mass renormalizations,
    \[
      \frac{m}{m^*} = \frac1{1 + F_1^s}, \quad
      \frac{m}{m_s^*} = \frac1{1 + F_1^a}.
    \]
    \item Show that, when the Fermi liquid is polarized, the masses of the
    ``up'' and ``down'' quasiparticles are now different, and given by
    \[
      \frac1{m_\sigma^*} = \frac1m\ab[\frac1{1 + F_1^s} + \frac M{1 + F_1^a}],
      \quad (\sigma = \uparrow, \downarrow),
    \]
    where the magnetization $M = n_\uparrow - n_\downarrow$ is the difference of
    ``up'' and ``down'' densities.
  \end{enumext}
\end{problem}
\begin{solution}\leavevmode
  \begin{enumext}
    \item \begin{proof} \let \qed \relax
    In the steady state, the distribution function is a shifted
    Fermi sphere, then
    \[
      \delta n_{\bm p\sigma} = -\frac{\partial n^0}{\partial \epsilon}
        (\bm p \cdot \bm u_\sigma),
    \]
    where $\bm u_\sigma$ is the drift velocity for $\sigma$.
    Substitute it and $\delta\epsilon_{\bm p\sigma}^{(0)}$ to get the full
    quasiparticle energy shift
    \[
      \delta\epsilon_{\bm p\sigma}
    = \delta\epsilon_{\bm p\sigma}^{(0)}
    + \sum_{\bm p'\sigma'} f_{\bm p\sigma,\bm p'\sigma'}
      \delta n_{\bm p'\sigma'}
    = -\frac{\bm p}{m} \cdot (\mathbf A_N + \sigma \mathbf W)
    - \sum_{\bm p'\sigma'} f_{\bm p\sigma,\bm p'\sigma'}
      \frac{\partial n^0}{\partial \epsilon'}
      (\bm p' \cdot \bm u_{\sigma'}).
    \]
    Due to isotropy, the interaction term simplifies. The Landau interaction
    function is decomposed into spin-symmetric and spin-antisymmetric parts
    \[
      f_{\bm p\sigma,\bm p'\sigma'}
    = f^s(\theta) \delta_{\sigma\sigma'} + f^a(\theta) (\sigma \sigma').
    \]
    where $\theta$ is the angle between $\bm p$ and $\bm p'$.
    Expanding in Legendre polynomials
    \[
      f^s(\theta) = \sum_l f_l^s P_l(\cos\theta), \quad
      f^a(\theta) = \sum_l f_l^a P_l(\cos\theta).
    \]
    The Landau parameters are defined as:
    \[
      F_l^s = N(0) f_l^s, \quad F_l^a = N(0) f_l^a,
    \]
    where $N(0)$ is the density of states at the Fermi level per spin.
    After evaluating the angular integrals, the energy shift becomes
    \[
      \delta\epsilon_{\bm p\sigma} = -\frac{\bm p}{m} \cdot (\mathbf A_N
    + \sigma \mathbf W) + \frac{1}{3}
      \ab[F_1^s (\bm p \cdot \bm u_{\sigma})
      + F_1^a \sigma (\bm p \cdot (\bm u_{\uparrow} - \bm u_{\downarrow}))].
    \]
    In the steady state, the effective force on quasiparticles vanishes, i.e.,
    $\delta\epsilon_{\bm p\sigma} = \bm p \cdot \bm u_\sigma$. Comparing with
    the formula above, we can obtain $\bm u_\sigma$
    \[
      \bm u_\sigma = -\frac{1}{m^*} \mathbf A_N
      - \sigma \frac{1}{m_s^*} \mathbf W.
    \]
    Then, comparing coefficients yields the desired relations
    \[
      \frac{m}{m^*} = \frac{1}{1 + F_1^s}, \quad
      \frac{m}{m_s^*} = \frac{1}{1 + F_1^a}.
    \]
    \end{proof}
    \item \begin{proof} When the system is magnetized
    ($M = n_\uparrow - n_\downarrow \neq 0$), the Fermi surfaces for spin-up and
    spin-down are different. The drift velocities become
    \[
      \bm u_\uparrow = \bm u + \bm u_s, \quad
      \bm u_\downarrow = \bm u - \bm u_s,
    \]
    where $\bm u$ is the charge drift velocity and $\bm u_s$ is the spin drift
    velocity. The energy shifts are
    \[
      \delta\epsilon_{\bm p\uparrow}
    = -\frac{\bm p}{m^*} \cdot \mathbf A_N
    - \frac{\bm p}{m_s^*} \cdot \mathbf W, \quad
      \delta\epsilon_{\bm p\downarrow}
    = -\frac{\bm p}{m^*} \cdot \mathbf A_N
    + \frac{\bm p}{m_s^*} \cdot \mathbf W.
    \]
    The effective masses for each spin species are then:
    \[
      \frac{1}{m_\sigma^*}
    = \frac{1}{m} \ab[ \frac{1}{1 + F_1^s} + \sigma \frac{M}{1 + F_1^a}],
      \qq{for $\sigma = \uparrow, \downarrow$,}
    \]
    where $ M $ is the magnetization.
    \end{proof}
  \end{enumext}
\end{solution}