% !TeX root = ../main.tex

\section{Homework \#6 [2025-10-16]}

\begin{problem}
  Derive the completeness relation Eq.~\eqref{negele2018quantum:1.123}
  \begin{equation}
    \int \prod_\alpha \frac{\d\phi_\alpha^* \d\phi_\alpha}{2\iu\pi}
    \upe^{-\sum_\alpha \phi_\alpha^*\phi_\alpha} \ketbra|\phi><\phi| = \identity
    \tag{1.123}
    \label{negele2018quantum:1.123}
  \end{equation}
  by integration. First consider
  one single-particle state $\alpha$ and let $\ket|n>$ denote the state with $n$
  particles in $\alpha$. Show that by writing in polar form
  $\phi = \rho \upe^{\iu\theta}$ one obtains
  \[
    \int \frac{\d\phi^* \d\phi}{2\pi\iu} \upe^{-\phi^*\phi} \ketbra|\phi><\phi|
  = \int \rho \frac{\d\rho \d\theta}{\pi} \upe^{-\rho^2}
    \sum_m \frac{(\rho\upe^{\iu\theta})}{\sqrt{m!}} \ket|m>
    \sum_n \frac{(\rho\upe^{-\iu\theta})}{\sqrt{n!}} \bra<n|
  = \sum_n \ketbra|n><n|.
  \]
  Now generalize to a set of single-particle states $\{\ket|\alpha>\}$ noting
  that the closure relation Eq. (1.89) may be written
  \[
    \sum_{\{n_\alpha\}}
    \ketbra|n_{\alpha_1} n_{\alpha_2} \ldots><n_{\alpha_1} n_{\alpha_2} \ldots|
  = \identity,
  \]
  where $\{n_\alpha\}$ denotes a complete set of occupation numbers.
\end{problem}
\begin{solution}
  Expand the coherent state $\ket|\phi>$ in terms of number state $\ket|n>$
  \[
    \ket|\phi> = \sum_{n=0}^\infty c_n \ket|n>.
  \]
  Here, we merge the normalization factor into $c_n$ to simplify, and it will be released at the end. Act $\hat a$ on the expand expression of $\ket|\phi>$, due to the property of annihilation operator
  \[
    \hat a \ket|\phi> = \sum_{n=0}^\infty c_n \hat a \ket|n>
  = \sum_{n=1}^\infty c_n \sqrt n \ket|n - 1>
  = \phi \sum_{n=0}^\infty c_n \ket|n>,
  \]
  the third term summing from $n = 1$ is due to $\hat a\ket|0> = 0$,
  which $n = 0$ make no sence. To uniform the lower limit of $n$,
  shift the summation index
  \[
    \hat a \ket|\phi>
  = \sum_{n=0}^\infty c_{n+1} \sqrt{n + 1} \ket|n>
  = \phi \sum_{n=0}^\infty c_n \ket|n>,
  \]
  then we obtain the expression of the expansion coefficient
  \[
    c_{n+1} \sqrt{n + 1} = \phi c_n, \qq{or}
    c_n = \frac\phi{\sqrt n} c_{n-1}
  = \frac\phi{\sqrt n} \frac\phi{\sqrt{n - 1}} c_{n-2} = \cdots
  = \frac{\phi^n}{\sqrt{n!}} c_0,
  \]
  thus we obtain the expansion of $\ket|\phi>$ in terms of number states
  \[
    \ket|\phi> = c_0 \sum_{n=0}^\infty \frac{\phi^n}{\sqrt{n!}} \ket|n>
  \xlongequal{\text{normalization}}
    \upe^{-|\phi|^2/2} \sum_{n=0}^\infty \frac{\phi^n}{\sqrt{n!}} \ket|n>.
  \]
  Writing $\phi$ in polar form $\phi = \rho \upe^{\iu\theta}$,
  we expand $\d\phi^* \d\phi$ via Wedge product
  \[
    \d\phi^* \wedge \d\phi
  = (\upe^{-\iu\theta} \d\rho - \iu\rho\upe^{-\iu\theta} \d\theta) \wedge
    (\upe^{\iu\theta} \d\rho + \iu\rho\upe^{\iu\theta} \d\theta)
  = 2\iu\rho\d\rho \wedge \d\theta,
  \]
  then $\d\phi^* \d\phi = 2\iu\rho\d\rho\d\theta$.
  Substitute the above expressions into the identity
  \[
  \identity = \int \frac{\d\phi^* \d\phi}{2\pi\iu} \upe^{-\phi^*\phi} \ketbra|\phi><\phi|
  = \int_0^{2\pi} \d\theta \int_0^\infty
    \frac{\rho\d\rho}{\pi} \upe^{-\rho^2} \sum_{m,n=0}^\infty
    \frac{\rho^{m+n} \upe^{(m-n)\iu\theta}}{\sqrt{m!n!}}
    \ketbra|m><n|.
  \]
  Evaluate the angular integral
  \[
    \int_0^{2\pi} \d\theta \upe^{(m-n)\iu\theta} = 2\pi \delta_{mn}.
  \]
  Evaluate the radial integral for $m = n$ (since there will be a $\delta_{mn}$ factor)
  \[
    \int_0^\infty \d\rho \upe^{-\rho^2} \rho^{2n+1}
  = \frac12\Gamma(n + 1) = \frac12n!.
  \]
  Substitute them into the integral
  \[
    I = \sum_{m,n=0} \frac1\pi \frac1{\sqrt{m!n!}}
        (2\pi\delta_{mn}) \ab(\frac12 n!) \ketbra|m><n|
      \xlongequal{\text{only $m=n$ survives}} \sum_n \ketbra|n><n|.
  \]
  When generalize to a set of complete single-particle basis $\{\ket|\alpha>\}$
  and build the full Fock space with occupation numbers $\{n_\alpha\}$, i.e.,
  \[
    \ket|\{\phi_\alpha\}> = \prod_\alpha \ket|\phi_\alpha>_\alpha,
  \]
  the product measure over modes gives the multi-mode resolution of identity.
  Concretely, for each mode $\alpha$ introduce a complex coordinate
  $\phi_\alpha$ and form the product integral
  \[
    \prod_\alpha \int \frac{\d\phi_\alpha^* \d\phi_\alpha}{2\pi\iu}
    \upe^{-\phi_\alpha^*\phi_\alpha} \otimes_\alpha
    \ketbra|\phi_\alpha><\phi_\alpha|,
  \]
  expanded into a Fock basis, each mode's angular integral enforces equality of
  occupation numbers in bra $\bra<|$ and ket $\ket|>$, each radial integral
  produces the factorial that cancels the $1/\sqrt{n!}$ factors,
  so the identity ends up with the sum over all occupation-number configurations
  \[
    \sum_{\{n_\alpha\}}
    \ketbra|n_{\alpha_1} n_{\alpha_2} \ldots><n_{\alpha_1} n_{\alpha_2} \ldots|
  = \identity_\text{Fock}.
  \]
  In other words, the modes are independent, and for each that has the
  completeness shown above.
\end{solution}
\newpage
\begin{problem}
  Generalize the properties of Grassmann variables demonstrated in Section 1.5
  for the pari of generators $\xi$, $\xi^*$ to the case $2p$ generators
  $\{\xi_1 \ldots \xi_p\xi_1^* \ldots \xi_p^*\}$. In particular, determine the
  general form of a function $f(\xi_\alpha)$ and an operator
  $A(\xi_\alpha^*, \xi_\alpha)$, show that $\pdv*{}{\xi_\alpha}$,
  $\pdv*{}{\xi_\beta}$, $\xi_\gamma$, and $\xi_\delta^*$ anticommute, determine
  if an analogous property holds for integration, find and verify an expression
  for the $p$-dimensional $\delta$-function $\delta^P(\bm \xi - \bm \xi')$, and
  generalise Eq.~\eqref{negele2018quantum:1.157}
  \begin{equation}
    \begin{aligned}
      \braket<f|g>
  & = \int \d\xi^* \d\xi (1 - \xi^*\xi) (f_0^* + f_1^*\xi) (g_0 + g_1\xi^*)\\
  & = -\int \d\xi^* \d\xi \xi^* \xi f_0^* g_0
    + \int \d\xi^* \d\xi \xi \xi^* f_1^*g_1 = f_0^* g_0 + f_1^* g_1
    \end{aligned}
    \tag{1.157} \label{negele2018quantum:1.157}
  \end{equation}
  for $\braket<f|g>$.
\end{problem}
\begin{solution}\leavevmode
  \begin{enumext}
    \item Function $f(\bm\xi) \equiv f(\xi_1,\ \xi_1,\ \ldots\,,~\xi_p)$ is a general
    element of the Grassmann algebra generated by
    $\{\xi_1,\ \xi_1,\ \ldots\,,~\xi_p\}$. Since $\xi_\alpha^2 = 0$,
    any function must be ``at most'' linear in each variable $\xi_\alpha$, and
    the function is a \emph{finite polynomial}.
    Apply Taylor expansion to $f(\xi)$, we can obtain its general form
    \[
      f(\xi) = f_0 + \sum_{i=1}^p f_i\xi_i
    + \sum_{i<j}^p f_{ij}\xi_i\xi_j + \cdots + f_{1\ldots p} \xi_1\cdots\xi_p
    = \sum_{n=0}^p \frac1{n!} \sum_{\alpha_1<\alpha_2<\cdots<\alpha_n}
      f_{\alpha_1\cdots\alpha_n} \xi_{\alpha_1\cdots\alpha_n},
    \]
    where the coefficients $f_{\{\alpha_i\}}$ are antisymmertic under
    interchange of any two indices.
    The Operator depends on $2p$ generators. Apply the same logic to the
    function, the operator is also a finite linear combination of monomials
    \[
      A(\xi^*, \xi) = \sum_{m = 0}^p \sum_{n = 0}^p
      \sum_{\substack
        {\alpha_1<\alpha_2<\cdots<\alpha_n\\\beta_1<\beta_2<\cdots<\beta_n}}
      \frac1{m!n!}
      A^{(\alpha_1,\alpha_2,\ldots,\alpha_m)}_{(\beta_1,\beta_2,\ldots,\beta_n)}
      \xi_{\alpha_1}^* \xi_{\alpha_2}^* \cdots \xi_{\alpha_m}^*
      \xi_{\beta_1}    \xi_{\beta_2}    \cdots \xi_{\beta_n},
    \]
    with coefficients $A_{\{\beta_j\}}^{\{\alpha_i\}}$ antisymmetric in the
    upper indices and antisymmetric in the lower indices.
    \item Act the partial derivatives with respect to Grassmann variables on a
    Grassmann variable, or a multiple of Grassmann functions, can be expressed as
    \[
      \pdif{\xi_\alpha}(\xi_\beta) = \delta_{\alpha\beta}, \qq{and}
      \pdif{\xi_\alpha} (AB) = (\pdif{\xi_\alpha} A)B
    + (-1)^{\epsilon_A} A (\pdif{\xi_\alpha} B),
    \]
    where $\epsilon_A = 0$ (even) or $1$ (odd) is the Grassmann parity of $A$.
    Then, we can obtain the canonical anticommutators
    \[
      [\xi_\alpha, \xi_\beta]_+ = 0, \qq{and}
      [\pdif{\xi_\alpha}, \pdif{\xi_\beta}]_+ = 0,
    \]
    and the mixed relation
    \[
      [\pdif{\xi_\alpha}, \xi_\beta]_+
    = \pdif{\xi_\alpha} \xi_\beta + \xi_\beta \pdif{\xi_\alpha}
    = \delta_{\alpha\beta}.
    \]
    With respect to the integration properties
    \[
      \int \d\xi_\alpha = 0, \quad \int \d\xi_\alpha \xi_\alpha = 1, \quad
      \d\xi_\alpha^{(*)} \d\xi_\beta^{(*)}
    = -\d\xi_\beta^{(*)} \d\xi_\alpha^{(*)}.
    \]
    With the chosen ordering the standard normalization is
    \[
      \d^p\bm\xi \equiv \d\xi_p \d\xi_{p-1} \cdots \d\xi_1, \quad
      \int \d^p\bm\xi_{\alpha_1} \xi_{\alpha_2} \cdots \xi_{\alpha_p}
    = (-1)^{\epsilon_{\xi_i}} \delta_{np}.
    \]
    \item Due to the integration properties and the \emph{Dirac measure}, the
    $p$-dimensional $\delta$-function satisfies
    \[
      \int \d^p\xi f(\bm\xi) \delta^p(\bm\xi - \bm\xi') = f(\bm\xi'),
    \]
    then, the $p$-dimensional $\delta$-function can be expressed as
    \[
      \delta^p(\bm\xi - \bm\xi') \equiv \prod_{\alpha=1}^p
      (\xi_\alpha - \xi_\alpha'),
    \]
    and it can be trivially verified by denoting
    $f(\bm\xi') = \sum_i f_i\xi_i'$.
    \[
      \int \d\xi_1 f(\xi_1,\ \xi_2,\ \ldots\,,~\xi_p)
        (\xi_1 - \xi_1') \cdots (\xi_p - \xi_p')
    = f(\xi_1',\ \xi_2,\ \ldots\,,~\xi_p)
        (\xi_2 - \xi_2') \cdots (\xi_p - \xi_p'),
    \]
    which it is similar for $\xi_2$, $\ldots\,$,~$\xi_p$.
    And after $p$ integrations, the RHS becomes
    $f(\xi_1',\ \xi_2',\ \ldots\,,~\xi_p') \equiv f(\bm\xi')$.
    \item Expand $f(\xi)$ and $g(\xi^*)$ as
    \[
      f(\xi) = \sum_i f_i\xi_i, \qq{and}
      g(\xi^*) = \sum_j g_j\xi_j^*,
    \]
    then, the inner product can be expressed as
    \[
      \braket<f|g> = \int \ab(\prod_{\alpha=1}^p \d\xi_\alpha^* \d\xi_\alpha)
      \ab[\prod_{\alpha=1}^p (1 - \xi_\alpha^*\xi_\alpha)] f^*(\xi) g(\xi^*)
      \equiv \sum_i^p f_i^* g_i.
    \]
    If $p = 2$, then we will generate Eq. \eqref{negele2018quantum:1.157}, i.e.,
    $\braket<f|g> = f_0^* g_0 + f_1^* g_1$.
  \end{enumext}
\end{solution}
\newpage
\begin{problem}
  Prove the closure relation Eq.~\eqref{negele2018quantum:1.123}
  \begin{equation}
    \int \prod_\alpha \frac{\d\phi_\alpha^* \d\phi_\alpha}{2\iu\pi}
    \upe^{-\sum_\alpha \phi_\alpha^*\phi_\alpha} \ketbra|\phi><\phi| = \identity
    \tag{1.123}
  \end{equation}
  to its fermionic form
  \[
    \identity = \int \d\mu(\xi) \upe^{-\sum_\alpha \xi_\alpha^* \xi_\alpha}
    \ketbra|\xi><\xi|
  \]
  for Fermions using Schur's lemma. As in the Boson case, $\ab[a_\alpha,
  \ket*\Big|\phi>\braket*\Big<\phi \|= \ab(\xi_\alpha - \pdv*{}{\xi_\alpha'})|\phi>\bra*\Big<\phi|]$,
  so one must show
  \begin{equation}
    \int \prod_\alpha \d\xi_\alpha^* \d\xi_\alpha
    \upe^{-\sum_\alpha \xi_\alpha^* \xi_\alpha}
    \ab(\xi_\alpha - \pdv*{}{\xi_\alpha^*}) A(\xi_\alpha, \xi_\alpha^*) = 0
    \tag{P6.1} \label{problem:6.1}
  \end{equation}
  for any $A$. First, prove Eq.~\eqref{problem:6.1} as it stands, establishing
  that the left-hand side of Eq.~\eqref{negele2018quantum:1.126}
  \begin{equation}
    \ab[a_\alpha,
      \int\prod_{\alpha'} \frac{\d\phi_{\alpha'}^* \d\phi_{\alpha'}}{2\pi\iu}
      \upe^{-\sum_{\alpha'}\phi_{\alpha'}^*\phi_{\alpha'}} \ketbra|\phi><\phi|
    ] =
    \int\prod_{\alpha'} \frac{\d\phi_{\alpha'}^* \d\phi_{\alpha'}}{2\pi\iu}
      \upe^{-\sum_{\alpha'}\phi_{\alpha'}^*\phi_{\alpha'}} \ketbra|\phi><\phi|
    = 0
    \tag{1.126} \label{negele2018quantum:1.126}
  \end{equation}
  must be proportional to unity, and evaluate the constant of proportionality
  by calculating the matrix element in the zero-particle state.

  Then note that Eq.~\eqref{problem:6.1} is a special case of integration by parts for
  Grassmann variables. Show that the general rule for integration by parts is
  \begin{equation}
    \int \prod_\alpha \d\xi_\alpha^* \d\xi_\alpha A(\xi_\alpha^*,\xi_\alpha)
    \frac{\vec\partial}
      {\partial\xi_\alpha} B(\xi_\alpha^*, \xi_\alpha) =
    \int \prod_\alpha \d\xi_\alpha^* \d\xi_\alpha A(\xi_\alpha^*,\xi_\alpha)
    \frac{\cev\partial}{\partial\xi_\alpha} B(\xi_\alpha^*, \xi_\alpha)
    \tag{P6.2} \label{problem:6.2}
  \end{equation}
  where $\frac{\cev\partial}
    {\partial\xi_\alpha}$ acts to the left and the variable $\xi_\alpha$ must
  be anticommuted to the right to be adjacent to the derivative. Note in
  particular that the sign in Eq.~\eqref{problem:6.2} is reversed from the usual
  relation for complex variables, and that expressions like
  $\int \prod_\alpha \d\xi_\alpha^* \d\xi_\alpha \pdv*{}{\xi_\alpha}
    [A(\xi_\alpha^*, \xi_\alpha)] B(\xi_\alpha^*, \xi_\alpha)$ do not reproduce
  the right-hand side.
\end{problem}
\begin{solution}
  We need to list some identities of the Grassmann variables.
  \begin{enumext*}[columns = 13]
    \item(2) $\xi^2 = 0$
    \item(2) $\xi = \xi^*$
    \item(2) $\xi^* \xi = -\xi \xi^*$
    \item(2) $\int 1 \d\xi = 0$
    \item(2) $\int \xi \d\xi = 1$
    \item(3) $\int \ab[\pdv{f(\theta)}\theta] \d\theta = 0$
  \end{enumext*}
  Then the exponential term and $A(\xi^*, \xi)$ can be expanded as
  \[
    \upe^{-\xi^*\xi} = 1 - \xi^*\xi, \quad \upe^{-\xi^*\xi} \xi = \xi, \qq{and}
    A(\xi^*, \xi) = A_{00} + A_{01}\xi + A_{10}\xi^* + A_{11}\xi\xi^*,
  \]
  where the other terms consist $\xi^2$, ${\xi^*}^2$, or $(\xi^*\xi)^2$ vanish.
  To prove Eq.~\eqref{problem:6.1}, consider
  \[
    \int \d\xi^* \d\xi \upe^{-\xi^*\xi} (\xi - \pdif{\xi^*}) A(\xi^*, \xi).
  \]
  The first term of the integral can be expanded as
  \[
    \int \d\xi^* \int \d\xi \upe^{-\xi^*\xi} \xi A(\xi, \xi^*)
  = \int \d\xi^* \int \d\xi  \xi (A_{00} + A_{01}\xi + A_{10}\xi^* + A_{11}\xi\xi^*)
  = A_{10},
  \]
  where only the term containing $\xi\xi^*$ survives.
  Since the derivative of two Grassmann terms satisfies
  \[
    \pdif{\xi^*}[\upe^{-\xi^*\xi} A(\xi, \xi^*)]
  = \pdif{\xi^*}(\upe^{-\xi^*\xi}) A(\xi, \xi^*)
  + \upe^{-\xi^*\xi} \pdif{\xi^*} A(\xi, \xi^*).
  \]
  Then the second term of the integral can be expanded as
  \begin{align*}
  & \int \d\xi^* \int \d\xi \upe^{-\xi^*\xi} \pdif{\xi^*} A(\xi, \xi^*)
= \xcancel{\int \d\xi^* \int \d\xi \pdif{\xi^*}[\upe^{-\xi^*\xi} A(\xi, \xi^*)]}
  - \int \d\xi^* \int \d\xi \pdif{\xi^*}(\upe^{-\xi^*\xi}) A(\xi, \xi^*)\\
= & - \int \d\xi^* \int \d\xi \pdif{\xi^*}(1 - \xi^*\xi) A(\xi, \xi^*)
= \int \d\xi^* \int \d\xi \xi
  (A_{00} + A_{01}\xi + A_{10}\xi^* + A_{11}\xi\xi^*)
= A_{10}.
  \end{align*}
  Combining the two terms, we have
  \[
    \int \d\xi^* \d\xi \upe^{-\xi^*\xi} (\xi - \pdif{\xi^*}) A(\xi^*, \xi)
  = A_{10} - A_{10} = 0.
  \]
  Concerning multi-mode: for a fixed label $\beta$, consider the total
  derivative $\pdif{\xi_\beta^*}$ to
  $\upe^{-\sum_\alpha\xi_\alpha^*\xi_\alpha} A(\xi_\alpha, \xi_\alpha^*)$.
  The same algebra gives
  \[
    \pdv*{\upe^{-\sum_\alpha\xi_\alpha^*\xi_\alpha} A(\xi_\alpha, \xi_\alpha^*)}{\xi_\beta^*}
  = -\xi_\beta \upe^{-\sum_\alpha \xi_\alpha^*\xi_\alpha}
  + \upe^{-\sum_\alpha \xi_\alpha^*\xi_\alpha} \pdif{\xi_\beta^*} A(\xi, \xi^*),
  \]
  and integrating over $\prod_\alpha \d\xi_\alpha^* \d\xi_\alpha$ yields zero.
  Thus, the multi-mode identity Eq.~\eqref{problem:6.1} follows.
  Then, we shall generate Eq.~\eqref{negele2018quantum:1.126}.
  Consider the single mode first, we calculate the ``kernel'' commutator
  $[\hat a, \ketbra|\xi><\xi|]$.
  \begin{enumext}
    \item $a \ket|\xi>$.
    Since $\ket|\xi>$ can be expanded as
    \[
      \ket|\xi> = \upe^{a^\dagger\xi} \ket|0> = (1 + a^\dagger\xi) \ket|0>.
    \]
    Left act $a$ on it, then we have
    \[
      a\ket|\xi> = a \ket|0> + aa^\dagger\xi\ket|0>
    = 0 + (1 + a^\dagger a)\xi\ket|0> = \xi\ket|0>.
    \]
    If we multiply $\xi$ to $\ket|\xi>$
    \[
      \xi\ket|\xi> = \xi (1 + a^\dagger\xi) \ket|0> = \xi \ket|0>,
    \]
    the $\xi^2$ term vanishes. So, we get the expression
    $a\ketbra|\xi><\xi| = \xi\ketbra|\xi><\xi|$.
    \item $\bra<\xi| a$.
    Since $\bra<\xi|$ can be expanded as
    \[
      \bra<\xi| = \upe^{a\xi^*} \bra<0| = \bra<0|(1 + a\xi^*).
    \]
    Right act $a$ on it, then we have
    \[
      \bra<\xi|a = \bra<0|(1 + a\xi^*) a = \bra<0|a + \bra<0| a\xi^*a
    = \bra<0| a,
    \]
    where $\bra<0|a = (a^\dagger\ket|0>)^\dagger = \bra<1|$,
    and $aa$ forms a zero operator, i.e., $aa = 0$ (for Fermions).
    If we right act $\pdif{\xi^*}$ on $\bra<\xi|$
    \[
      \bra<\xi| \cev\partial_{\xi^*} = (1 + a\xi^*) \bra<0| \cev\partial_{\xi^*}
    = \bra<0|a,
    \]
    So, we get the expression $\bra<\xi|a = \bra<\xi| \cev\partial_{\xi^*}$,
    and we have
    \[
      (\ketbra|\xi><\xi|) \cev\partial_{\xi^*}
    = (\ket|\xi>\cev\partial_{\xi^*})\bra<\xi| +
      \ketbra|\xi>(<\xi|\cev\partial_{\xi^*})
    = ((1 + a^\dagger\xi)\ket|0>\cev\partial_{\xi^*})\bra<\xi| +
      \ketbra|\xi>(<\xi|\cev\partial_{\xi^*})
    = \ketbra|\xi>(<\xi|\cev\partial_{\xi^*}).
    \]
    \item Prove the relation between left and right derivatives.
    Denote
    \[
      \d\mu(\xi) \equiv \prod_\alpha \d\xi_\alpha^*\xi_\alpha.
    \]
    Due to the identities of the derivative of Grassmann variables
    \[
      \pdv*[fun]{AB}\xi = \ab(\pdv A\xi)B + (-1)^{\epsilon_A} A\ab(\pdv B\xi),
    \]
    where $\epsilon_A$ is the parity of $A$.
    Integral over the total derivative, and using the partial integration
    formula of Grassmann variables
    \begin{align*}
      0 \equiv \int \d\mu \vec\partial_\xi(AB)\xi
    & = \int \d\mu (\vec\partial_\xi A) B
    + (-1)^{\epsilon_A} \int \d\mu A (\vec\partial_\xi B),\\
      \int \d\mu A (\vec\partial_\xi B)
    & = -(-1)^{\epsilon_A} \int \d\mu (\vec\partial_\xi A) B.
    \end{align*}
    Due to the property of the Grassmann variable,
    the left-acting derivative to $A$ can be expanded as
    \[
      A\cev{\partial_\xi} = -(-1)^{\epsilon_A} \vec\partial_\xi A,
    \]
    by combining the two expressions above, we can obtain
    \[
      \int \d\mu A(\vec\partial_\xi B) = \int \d\mu A\cev\partial_\xi B.
    \]
    Also for multi-mode, which we proved Eq.~\eqref{problem:6.2}.
  \end{enumext}
  Then, the ``kernel'' commutator becomes
  \[
    [\hat a, \ketbra|\xi><\xi|] = a\ketbra|\xi><\xi| - \ketbra|\xi><\xi|a
  = \xi\ketbra|\xi><\xi| - (\ketbra|\xi><\xi|) \cev\partial_{\xi^*}
  = (\xi - \vec\partial_{\xi^*}) \ketbra|\xi><\xi|,
  \]
  and the commutator in Eq.~\eqref{negele2018quantum:1.126} can be written as
  \[
    \ab[a, \int \d\mu \upe^{\xi^* \xi} \ketbra|\xi><\xi|]
  = \int \d\mu \upe^{\xi^*\xi} (\xi - \vec\partial_{\xi^*}) \ketbra|\xi><\xi|
  \xlongequal{\text{Eq.~\eqref{problem:6.1}}} 0,
  \]
  and it also satisfies multi-mode, which we proved Eq.~
  \eqref{negele2018quantum:1.126}. Due to Schur's lemma, the operator
  $\mathcal O \equiv \int \d\mu \upe^{\xi^* \xi} \ketbra|\xi><\xi|$
  commutes to $a$, then it is proportional to unity,
  i.e., $\mathcal O = C\mathbbm 1$. To determine $C$, calculate the vacuum
  expectation of $\mathcal O$
  \[
    C \equiv \braket<0|\mathcal O|0> = \int \d\mu(\xi)
    \upe^{-\sum_\alpha \xi_\alpha^* \xi} \braket<0|\xi> \braket<\xi|0>.
  \]
  Since the expansion of $\ket|\xi>$, $\bra<\xi|$, and $\upe^{-\xi^*\xi}$,
  we have
  \[
    \braket<0|\xi> = \braket<\xi|0> = 1, \qq{and}
    \int \d\xi^*\d\xi \upe^{-\xi^*\xi} = \int \d\xi^*\d\xi (1 - \xi^*\xi) = 1.
  \]
  Therefore
  \[
    C = 1, \quad \mathcal O = \mathbbm 1, \qq{and}
    \int\d\mu(\xi) \upe^{-\sum_\alpha \xi_\alpha^* \xi} \ketbra|\xi><\xi|
  = \identity,
  \]
  which we proved the fermionic form for Eq.~\eqref{negele2018quantum:1.123}.
\end{solution}