% !TeX root = ../main.tex

\section{Homework \#1 [2025-09-02]}

\begin{problem}
  Prove the following equations
  \begin{gather*}
    [\hat d_n, \hat d_m^\dagger]_{\mp} = \braket<n|m> = \delta_{nm},\\
    \hat d_n\ket|n> = \ket|0>.
  \end{gather*}
\end{problem}
\begin{solution}
\begin{proof}
  Reviewing the definitions of $\hat d_n^\dagger$ and $\hat d_n$
  \[
    \hat d_n = \int \d x \varphi_n^*(x) \hat\psi(x),\qq{and}
    \hat d_n^\dagger = \int \d x \varphi_n(x) \hat\psi^\dagger(x),
  \]
  then substitute them into the (anti)commutator
  \begin{align*}
    [\hat d_n, \hat d_m^\dagger]_{\mp} & =
    \ab(\int \d x \varphi_n^*(x) \hat\psi(x))^\dagger
    \ab(\int \d x' \varphi_m(x') \hat\psi(x')) \mp
    \ab(\int \d x' \varphi_m(x') \hat\psi(x'))
    \ab(\int \d x \varphi_n^*(x) \hat\psi(x))^\dagger\\
& = \int \d x \int \d x' \varphi_n^*(x) \varphi_m(x')
    [\hat\psi^\dagger(x) \hat\psi(x') \mp \hat\psi(x') \hat\psi^\dagger(x)]\\
& = \int \d x \int \d x' \varphi_n^*(x) \varphi_m(x')
    [\hat\psi^\dagger(x), \hat\psi(x')]_\mp.
  \end{align*}
  Since the (anti)commutator of the field operator is
  \[
    [\hat\psi^\dagger(x), \hat\psi(x')]_\mp = \delta(x - x'),
  \]
  integrate over $x'$, and then recognize the inner product expression.
  Since the functions $\varphi_n$, $\varphi_m$ are orthonormal, we arrive at
  \[
    [\hat d_n, \hat d_m^\dagger]_{\mp}
  = \int \d x \mathemph{\int \d x'} \varphi_n^*(x)
              \mathemph[\int]{\varphi_m(x') \delta(x - x')}
  = \int \d x \varphi_n^*(x) \varphi_m(x)
  = \braket<\varphi_n|\varphi_m> = \delta_{nm}.
  \]
  The integration above used the \emph{Dirac measure} of the $\delta$-function.
\end{proof}
\begin{proof}
  The one-particle ket $\ket|n>$ can be expanded in the position-spin kets
  using the identity
  \[
    \ket|n> = \int \d x \ket|x> \braket<x|n> = \int \d x \varphi_n(x) \ket|x>
  = \int \d x \varphi_n(x) \hat\psi^\dagger(x) \ket|0>.
  \]
  Since $\hat\psi(x) \ket|0> = 0$,
  that is \emph{the annihilation operator kills the vacuum},
  then we have
  \[
    \hat\psi(x) \hat\psi^\dagger(x') \ket|0>
  = [\hat\psi(x), \hat\psi^\dagger(x')]_\mp\ket|0>
  = \delta(x - x') \ket|0>,
  \]
  then substitute the relation above, the expression of $\hat d_n$
  and the expansion of $\ket|n>$ into $\hat d_n \ket|n>$
  \begin{align*}
    \hat d_n \ket|n> & =
    \int \d x \varphi_n^*(x) \hat\psi(x)
    \int \d x' \varphi_n(x') \hat\psi^\dagger(x') \ket|0>
  = \int \d x \mathemph{\int \d x'} \varphi_n^*(x)
              \mathemph[\int]{\varphi_n(x') \delta(x - x')} \ket|0>\\
& = \int \d x \varphi_n^*(x) \varphi_n(x) \ket|0>
  = \int \d x |\varphi_n(x)|^2 \ket|0> = \ket|0>.
  \end{align*}
  The integration above used \emph{the sifting property of the $\delta$-function},
  and the normalization of $\varphi_n(x)$.
\end{proof}
\end{solution}

\begin{problem}
  Prove the inverse relations (3.6.4)%
  % \footnote{\citet[see][eq. (1.60)]{4_stefanucci2013nonequilibrium}}
  i.e., the inverse relations of
  \[
    \hat\psi(\bm x) = \sum_n \varphi_n(\bm x)\hat d_n, \qq{and}
    \hat\psi^\dagger(\bm x) = \sum_n \varphi_n^*(\bm x)\hat d_n^\dagger,
  \]
  meaning expressing $\hat d_n$ and $\hat d_n^\dagger$ by $\hat\psi(x)$.
\end{problem}
\begin{solution}
  To express $\hat d_n$ (and $\hat d_n^\dagger$),
  we can use the orthogonality
  $\int \d x \varphi_m^*(\bm x) \varphi_n(\bm x) = \delta_{mn}$
  to eliminate itself on the right side.
  Left multiply both sides of $\varphi_m^*(\bm x)$ by $\hat \psi(\bm x)$,
  and then integrate over $x$
  \[
    \int \d x \varphi_m^*(\bm x) \hat\psi(\bm x)
  = \int \d x \varphi_m^*(\bm x) \ab(\sum_n \varphi_n(\bm x)\hat d_n).
  \]
  Since the integral and sum actions are exchangeable in this case,
  then we arrive at
  \[
    \int \d x \varphi_m^*(\bm x) \hat\psi(\bm x)
  = \sum_n \hat d_n \int \d x \varphi_m^*(\bm x) \varphi_n(\bm x)
  = \sum_n \delta_{mn} \hat d_n = \hat d_m.
  \]
  Similarly, for the creation operator $\hat d_n^\dagger$,
  repeat the actions above to $\hat\psi(\bm x)^\dagger$ as following
  \[
    \int \d x \varphi_m(\bm x) \hat\psi(\bm x)^\dagger
  = \sum_n \hat d_n^\dagger \int \d x \varphi_m(\bm x) \varphi_n(\bm x)^*
  = \sum_n \delta_{mn} \hat d_n^\dagger = \hat d_m^\dagger.
  \]
  To summarize,
  the inverse relations of $\hat \psi(\bm x)$ and $\hat \psi(\bm x)^\dagger$
  for mode $n$ are given by
  \[
    \hat d_n = \int \d x \varphi_n^*(\bm x) \hat\psi(\bm x), \qq{and}
    \hat d_n^\dagger = \int \d x \varphi_n(\bm x) \hat\psi(\bm x)^\dagger.
  \]
\end{solution}

\begin{problem}
  Let $\ket|n> = \ket|\bm p\tau>$ be a momentum-spin ket so that
  $\braket<\bm x|\bm p\tau> = \upe^{\iu\bm p\cdot r} \delta_{\sigma\tau}$.%
  \footnote{\citep[see][eq. (1.10)]{4_stefanucci2013nonequilibrium}}
  \[
    \braket<\bm x|\bm p\sigma'> = \delta_{\sigma\sigma'}\braket<\bm r|\bm p>
    \qq{with}
    \braket<\bm r|\bm p> = \upe^{\iu\bm p \cdot \bm r}.
  \]
  Show that the (anti)commutation relation in (3.6.2) then reads
  \[
    [\hat d_{\bm p\tau}, \hat d_{\bm p',\tau'}^\dagger]_\mp
  = (2\pi)^3 \delta^3(\bm p - \bm p') \delta_{\tau\tau'},
  \]
  and that the expansion (3.6.4) of the field operators in terms of the $\hat d$-operators is
  \[
    \hat\psi(\bm x) = \int \frac{\d^3\bm p}{(2\pi)^3}
      \upe^{\iu\bm p\cdot\bm r} \hat d_{\bm p\sigma}, \qq{and}
    \hat\psi^\dagger(\bm x) = \int \frac{\d^3\bm p}{(2\pi)^3}
      \upe^{-\iu\bm p \cdot\bm r} \hat d_{\bm p\sigma}^\dagger.
  \]
\end{problem}
\begin{solution}
\begin{proof}
  When $\ket|n> = \ket|\bm p\tau>$,
  $\varphi_n(x)$ becomes $\varphi_{\bm p\tau} (\bm x) =
  \braket<\bm x|\bm p\tau> = \delta_{\sigma\tau} \upe^{\iu\bm p\cdot\bm r}$.
  Using the combined coordinate $\bm x \equiv (\bm r, \sigma)$ that
  provides a complete description of a particle's state:
  continuous spatial position $\bm r$ and
  discrete spin projection quantum number $\sigma$.
  The expression of the operator $\hat d_{\bm p\tau}$ in terms of
  $\hat\psi(\bm r, \sigma)$ should be
  \emph{not only integrate over $\bm r$, but also sum over spin $\sigma$}.
  That is
  \begin{align*}
    \hat d_{\bm p\tau} &
  = \int \d^3 \bm x
    \varphi_{\bm p\tau}^*(\bm x)  \hat\psi(\bm x)
  = \int \d^3 \bm r \sum_\sigma \delta_{\sigma\tau}
    \upe^{-\iu \bm p\cdot\bm r} \hat\psi(\bm r,\sigma)
  = \int \d^3 \bm r \upe^{-\iu \bm p\cdot\bm r} \hat\psi(\bm r,\tau),\\
    \hat d_{\bm p\tau}^\dagger &
  = \int \d^3 \bm x
    \varphi_{\bm p\tau}(\bm x)  \hat\psi(\bm x)^\dagger
  = \int \d^3 \bm r \sum_\sigma \delta_{\sigma\tau}
    \upe^{\iu \bm p\cdot\bm r} \hat\psi(\bm r,\sigma)^\dagger
  = \int \d^3 \bm r \upe^{\iu \bm p\cdot\bm r} \hat\psi(\bm r,\tau)^\dagger.
  \end{align*}
  Promoting $[\hat\psi^\dagger(x), \hat\psi(x')]_\mp = \delta(x - x')$ to
  ``4D'' (means 3D coordinate $r$ + 1D spin $\sigma$ (or the variable $\tau$))
  \[
    [\hat\psi^\dagger(\bm x), \hat\psi(\bm x')]_\mp
  = \delta^3(\bm r - \bm r') \delta_{\tau\tau'},
  \]
  we get a ``4D'' $\delta$-result.
  Then substitute $\hat d_{\bm p\tau}$ and $\hat d_{\bm p\tau}^\dagger$
  into the commutator
  \begin{align*}
    [\hat d_{\bm p\tau}, \hat d_{\bm p',\tau'}^\dagger]_\mp
& = \int \d^3\bm r \int \d^3\bm r'
    \upe^{-\iu (\bm p\cdot\bm r-\bm p'\cdot\bm r')}
    [\hat\psi(\bm r,\tau), \hat\psi(\bm r',\tau')^\dagger]_\mp\\
& = \int \d^3\bm r \int \d^3\bm r'
    \upe^{-\iu (\bm p\cdot\bm r-\bm p'\cdot\bm r')}
    \delta^3(\bm r - \bm r') \delta_{\tau\tau'}\\
& = \int \d^3\bm r \upe^{-\iu (\bm p\cdot\bm r-\bm p'\cdot\bm r)}
    \delta_{\tau\tau'}
  = (2\pi)^3 \delta^3(\bm p - \bm p') \delta_{\tau\tau'}.
  \end{align*}
  The last line of the above derivation used
  the normalization of the plane waves, with a factor of $(2\pi)^3$.
\end{proof}
\begin{proof}
  Similar to Problem 1.1,
  we need to left multiply both sides by $\varphi_{p\tau}(\bm x)$,
  sum over $\tau$, and integrate over $\bm p$ to $\hat d_{\bm p\tau}(\bm x')$
  (In order to arrive at $\hat\psi(\bm x)$,
  a normalization factor $1/{(2\pi)^3}$ is applied)
  \[
    \sum_\tau \int \frac{\d^3\bm p}{(2\pi)^3}
      \varphi_{p\tau}(\bm x) \hat d_{\bm p\sigma}
  = \mathemph{\sum_\tau \int \frac{\d^3\bm p}{(2\pi)^3}
      \varphi_{p\tau}(\bm x)}
    \int \d^3 \bm x'
      \mathemph[\sum_\tau \frac{\d^3}{()^3}]
        {\varphi_{\bm p\tau}^*(\bm x')} \hat\psi(\bm x').
  \]
  \begin{framed}
    We arrive at the identity of $\varphi_{p\tau}(\bm x)$.
    Now, calculate it.
    \begin{align*}
      \text{Identity} & = \sum_\tau \int \frac{\d^3\bm p}{(2\pi)^3}
        \varphi_{p\tau}(\bm x) \varphi_{\bm p\tau}^*(\bm x')
    = \sum_\tau \int \frac{\d^3\bm p}{(2\pi)^3}
      \ab\big(\upe^{\iu\bm p\cdot\bm r} \delta_{\sigma\tau})
      \ab\big(\upe^{-\iu\bm p\cdot\bm r'} \delta_{\sigma'\tau})\\
  & = \sum_\tau \int \frac{\d^3\bm p}{(2\pi)^3}
      \upe^{\iu\bm p\cdot(\bm r-\bm r')} \delta_{\sigma\tau}\delta_{\sigma'\tau}
    = \int \frac{\d^3\bm p}{(2\pi)^3}
      \upe^{\iu\bm p\cdot(\bm r-\bm r')} \delta_{\sigma\sigma'}
    = \delta^3(\bm r - \bm r') \delta_{\sigma\sigma'},
    \end{align*}
    When handling $\delta_{\sigma\tau}\delta_{\sigma'\tau}$,
    we can consider choosing all of the $\tau$s (points)
    that equal to $\sigma$ and $\sigma'$ at the same time,
    it is equal to find filtering all the points on the axis that satisfy
    $\sigma = \sigma'$, that is
    $\sum_\tau \delta_{\sigma\tau} \delta_{\sigma'\tau} = \delta_{\sigma\sigma'}$.
  \end{framed}
  Now, return to the main topic.
  Substituting the identity and expand $\int \d^3 \bm x$
  \[
    \sum_\tau \int \frac{\d^3\bm p}{(2\pi)^3}
      \varphi_{p\tau}(\bm x) \hat d_{\bm p\sigma}
  = \sum_{\sigma'} \int \d^3 \bm r' \hat\psi(\bm r', \sigma')
    \delta^3(\bm r' - \bm r) \delta_{\sigma\sigma'}
  = \hat\psi(\bm r,\sigma) = \hat\psi(\bm x),
  \]
  that is
  \[
    \hat\psi(\bm x) = \sum_\tau \int \frac{\d^3\bm p}{(2\pi)^3}
      (\upe^{\iu\bm p\cdot\bm r} \delta_{\sigma\tau}) \hat d_{\bm p\sigma}
  = \int \frac{\d^3\bm p}{(2\pi)^3} \upe^{\iu\bm p\cdot\bm r} \hat d_{p\sigma}.
  \]
  Similarly to $\hat\psi^\dagger(\bm x)$. To summarize
  \[
    \hat\psi(\bm x) = \int \frac{\d^3\bm p}{(2\pi)^3}
      \upe^{\iu\bm p\cdot\bm r} \hat d_{\bm p\sigma}, \qq{and}
    \hat\psi^\dagger(\bm x) = \int \frac{\d^3\bm p}{(2\pi)^3}
      \upe^{-\iu\bm p \cdot\bm r} \hat d_{\bm p\sigma}^\dagger.
  \]
\end{proof}
\end{solution}