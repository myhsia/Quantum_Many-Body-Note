% !TeX root = ../main.tex

\section{Homework \#7 [2025-10-22]}

\begin{problem}
  Investigate a simple model Hamiltonian for BCS superconductivity,
  \[
    \mathcal H_k = t(k) (
      a_{k\uparrow}^\dagger a_{k\uparrow} +
      a_{-k\downarrow}^\dagger a_{-k\downarrow}) - \Delta(
      a_{k\uparrow}^\dagger a_{-k\downarrow}^\dagger +
      a_{-k\downarrow} a_{k\uparrow}).
  \]
  \begin{enumext}
    \item Find the energy eigenvalues of $\mathcal H_k$.
    \item Give the corresponding eigenstates.
    \item Give the possible excitation energies of the system.
    \item Identify the excitation gap, i.e. the minimum of the excitation
    energies. Consider the simplest 1D case: assuming $t(k) = -2t\cos k - \mu$.
  \end{enumext}
\end{problem}
\begin{solution}\leavevmode
  \begin{enumext}
    \item We consider the two big terms of the Hamiltonian, respectively.
    Working in the 4D Fock space, there are 4 identities
    \begin{enumext*}[columns = 14]
      \item(3) $\ket|0> \equiv \ket|0_{k\uparrow}, 0_{-k\downarrow}>$
      \item(3) $\ket|k\uparrow> \equiv a_{k\uparrow}^\dagger \ket|0>$
      \item(4) $\ket|-k\downarrow> \equiv a_{-k\downarrow}^\dagger \ket|0>$
      \item(4) $\ket|k\uparrow, -k\downarrow>
                \equiv a_{k\uparrow}^\dagger a_{-k\downarrow}^\dagger \ket|0>$
    \end{enumext*}
    Now, we shall compute how each of the 4 operators in $\mathcal H_k$ acts on
    the basis.
    \begin{enumext}
      \item Number term
      \[
        \begin{array}{*3{l@{~}c@{~}l@{\qquad}}l@{~}c@{~}l}
          a_{k\uparrow}^\dagger a_{k\uparrow} \ket|k\uparrow>   & =
        & \ket|k\uparrow>, &
          a_{k\uparrow}^\dagger a_{k\uparrow}
          \{\ket|0>,\ \ket|-k\downarrow>,\ \ket|k\uparrow, -k\downarrow>\} & =
        & \{0\},\\
          a_{-k\downarrow}^\dagger a_{-k\downarrow} \ket|k\uparrow>   & =
        & \ket|-k\downarrow>, &
          a_{-k\downarrow}^\dagger a_{-k\downarrow}
          \{\ket|0>,\ \ket|-k\downarrow>,\ \ket|k\uparrow, -k\downarrow>\} & =
        & \{0\}.
        \end{array}
      \]
      \item Pairing term
      \[
        \begin{array}{l@{~}c@{~}l@{\qquad}l@{~}c@{~}l}
          a_{k\uparrow}^\dagger a_{-k\downarrow}^\dagger \ket|0> & =
        & \ket|k\uparrow, -k\downarrow>, & 
          a_{k\uparrow}^\dagger a_{-k\downarrow}^\dagger
          \{\ket|k\uparrow>,\ \ket|-k\downarrow>,\
            \ket|k\uparrow, -k\downarrow>\} & = & \{0\},\\
          a_{-k\downarrow} a_{k\uparrow} \ket|k\uparrow, -k\downarrow>
         & = & \ket|0>, &
          a_{-k\downarrow} a_{k\uparrow}
          \{\ket|0>,\ \ket|k\uparrow>,\ \ket|k\downarrow>\} & = & \{0\}.
        \end{array}
      \]
    \end{enumext}
    \begin{minipage}[t]{.48\linewidth}
    The number term acts diagonally, giving
    \begin{center}
      \begin{tabular}{*5l}
        \toprule
        State & $\ket|0>$ & $\ket|k\uparrow>$ &
        $\ket|-k\downarrow>$ & $\ket|k\uparrow, -k\downarrow>$\\
        \midrule
        Energy& $0$ & $t(k)$ & $t(k)$ & $2t(k)$\\
        \bottomrule
      \end{tabular}
    \end{center}
    \end{minipage}
    \hspace*\fill
    \begin{minipage}[t]{.5\linewidth}
      Pairing term couples only $\ket|0>$ and
      $\ket|k\uparrow, -k\downarrow>$
      \begin{align*}
      & -\Delta(
        a_{k\uparrow}^\dagger a_{-k\downarrow}^\dagger +
        a_{-k\downarrow} a_{-k\downarrow}) \ket|0>
      = -\Delta \ket|k\uparrow, -k\downarrow>,\\
      & -\Delta(
        a_{k\uparrow}^\dagger a_{-k\downarrow}^\dagger +
        a_{-k\downarrow} a_{-k\downarrow}) \Delta \ket|k\uparrow, -k\downarrow>
      = -\ket|0>.
      \end{align*}
    \end{minipage}

    Since some of the actions act on, or create / annihilate the number of
    particles are odd, and other actions are even, so we can separate the
    Hamiltonian into two independent blocks
    \[
      \mathcal H_k =
      \begin{cases*}
        \{\ket|k\uparrow>,\ \ket|-k\downarrow>\}, & Odd sector,\\
        \{\ket|0>,\ \ket|k\uparrow,\ -k\downarrow>\}, & Even sector,
      \end{cases*}
    \]
    where
    \[\begin{array}{l@{~}c@{~}l@{~}c@{~}l}
      \mathcal H_\text{odd} & = &
      \begin{pmatrix}
        \braket<k\uparrow|\mathcal H_k|k\uparrow> &
        \braket<k\uparrow|\mathcal H_k|-k\downarrow>\\
        \braket<-k\downarrow|\mathcal H_k|k\uparrow> &
        \braket<-k\downarrow|\mathcal H_k|-k\downarrow>
      \end{pmatrix} & = &
      \begin{pmatrix}
        t(k) & 0\\
        0 & t(k)
      \end{pmatrix},\\
      \mathcal H_\text{even} & = &
      \begin{pmatrix}
        \braket<0|\mathcal H_k|0> &
        \braket<0|\mathcal H_k|k\uparrow, -k\downarrow>\\
        \braket<k\uparrow, -k\downarrow|\mathcal H_k|0> &
        \braket<k\uparrow, -k\downarrow|\mathcal H_k|k\uparrow, -k\downarrow>
      \end{pmatrix} & = &
      \begin{pmatrix}
        0 & -\Delta\\ -\Delta & 2t(k)
      \end{pmatrix}.
    \end{array}
    \]
    It is trivial that in the odd sector, both states have energy
    $E_\text{odd} = t(k)$, corresponding to 2nd degeneracy.
    In the even sector, we need to solve the eigenvalue equation
    \[
      \det|\mathcal H_\text{even} - \lambda| = 0.
    \]
    Hence, we have the eigenvalues for the even sector
    \[
      \lambda_1 = t(k) + \sqrt{t^2(k) + \Delta^2}, \quad
      \lambda_2 = t(k) - \sqrt{t^2(k) + \Delta^2}.
    \]
    In summary, the eigenvalues of $\mathcal H_k$ are
    \[
      E_\text{odd} = t(k), \quad
      E_\text{ground} = t(k) - \sqrt{t^2(k) + \Delta^2},\
      (\text{Degeneracy} = 2) \quad
      E_\text{excited} = t(k) + \sqrt{t^2(k) + \Delta^2}.
    \]
    \item Since in the odd sector, the Hamiltonian is diagonal, so it is trivial
    that the eigenstates in the odd sector are
    \[
      \psi_\text{odd 1} = \ket|k\uparrow>, \quad
      \psi_\text{odd 2} = \ket|-k\downarrow>.
    \]
    Concerning the even sector. Denote $E_k = \sqrt{t^2(k) + \Delta^2}$ for
    simplification. Since in the 4D Fock space, the basis should be orthonormal.
    So, assume the two eigenstates in the even sector to be
    \[
      \psi_\text{ground} = a_1\ket|0> + b_1\ket|k\uparrow, -k\downarrow>, \quad
      \psi_\text{excited} = a_2\ket|0> + b_2\ket|k\uparrow, -k\downarrow>.
    \]
    Since the orthonormality
    $\braket<\psi_\text{even 1}|\psi_\text{even 2}> = 0$, then we can derive
    \[
      \frac{a_1}{b_1} = -\frac{b_2}{a_2}.
    \]
    Let $a_1 = a_k$, $b_1 = b_k$. Then, the two eigenstates in the even sector
    can be expressed as
    \[
      \psi_\text{ground}  = a_k\ket|0> + b_k\ket|k\uparrow, -k\downarrow>, \quad
      \psi_\text{excited} = b_k\ket|0> - a_k\ket|k\uparrow, -k\downarrow>.
    \]
    where $a_k$ and $b_k$ should satisfies the normalization
    $a_k^2 + b_k^2 = 1$.
    To obtain $a_k$ and $b_k$, substitute $\psi_\text{ground}$ into the
    eigenequation
    \[
      \begin{pmatrix}
        0 & -\Delta\\ -\Delta & 2t(k)
      \end{pmatrix}
      \begin{pmatrix}
        a_k \\ b_k
      \end{pmatrix} =
      E_\text{ground}
      \begin{pmatrix}
        a_k \\ b_k
      \end{pmatrix}.
    \]
    Then we will obtain
    \[
      a_k^2 = \frac12\ab(1 + \frac{t(k)}{E_k}), \quad
      b_k^2 = \frac12\ab(1 - \frac{t(k)}{E_k}).
    \]
    So, the eigenstates in the even sector are
    \begin{align*}
      \psi_\text{ground} &
    = \sqrt{\frac12\ab(1 + \frac{t(k)}{E_k})} \ket|0>
    + \sqrt{\frac12\ab(1 - \frac{t(k)}{E_k})} \ket|k\uparrow, -k\downarrow>,\\
      \psi_\text{excited}&
    = \sqrt{\frac12\ab(1 - \frac{t(k)}{E_k})} \ket|0>
    - \sqrt{\frac12\ab(1 + \frac{t(k)}{E_k})} \ket|k\uparrow, -k\downarrow>.
    \end{align*}
    \item Since the energy eigenvalues of $\mathcal H_k$ are
    \[
      \{t(k),\ t(k),\ t(k) - E_k,\ t_k + E_k\}.
    \]
    It is trivial that the ground energy is $E_0 = t(k) - E_k$, and other there
    energies corresponding states are all excited states. So, the excitation
    energies of the system are
    \[
      \Delta E_\text{1qp} = t(k) - (t(k) - E_k) = E_k, \quad
      \Delta E_\text{2qp} = (t(k) + E_k) - (t(k) - E_k) = 2E_k.
    \]
    (Note: ``qp'' means ``quasi-particle'').
    \item It is trivial that the minimum of the excitation energies is
    \[
      \Delta E_{\min} = \Delta E_{1qp} = E_k = \sqrt{t^2(k) + \Delta^2}.
    \]
    Since $t(k) = -2t\cos k - \mu$, so the minimum occurs when $t(k) = 0$,
    i.e.,
    \[
      k = \cos^{-1}\ab(-\frac{\mu}{2t}) = \pi - \cos^{-1}\ab(\frac{\mu}{2t}).
    \]
    Then, the minimum excitation energy in the simplest 1D case is
    \[
      \Delta E_{\min} = \Delta.
    \]
  \end{enumext}
\end{solution}
\newpage
\begin{problem}
  \begin{enumext}\leavevmode
    \item Show that the excitations of a BCS superconductor are generated from
    Exercise 7.1 by the operators
    \[
      \rho_{k\uparrow}^\dagger
    = u_k a_{k\uparrow}^\dagger - v_k a_{-k\downarrow}; \quad
      \rho_{-k\downarrow}^\dagger
    = u_k a_{-k\downarrow}^\dagger + v_k a_{k\uparrow}.
    \]
    The coefficients $u_k$ and $v_k$ are defined as
    \[
      u_k^2 = \frac12 \ab\{1 + \frac{t(k)}{[t^2(k) + \Delta^2]^{1/2}}\}, \quad
      v_k^2 = 1 - u_k^2.
    \]
    \item Prove that these operators are purely Fermionic operators.
    \item Compute the commutator
    \[
      [\mathcal H^*, \rho_{k\uparrow}^\dagger]_-.
    \]
    How is this result to be interpreted?
    \item Formulate and solve the equation of motion of the retarded Green's
    function
    \[
      \hat G_{k\uparrow}^\text{ret}(E)
    = \ab*<\braket<\rho_{k\uparrow};\rho_{k\uparrow}^\dagger>>_E^\text{ret}.
    \]
    \item Compute Green's functions of the original fermions $a_k$s and
    $a_k^\dagger$s, consider all non-zero combinations. You can express them
    via linear transformations between $\rho_k$s and $a_k$s.
  \end{enumext}
\end{problem}
\begin{solution}\leavevmode
  \begin{enumext}
    \item \begin{proof}
      Obviously, the given coefficients $u_k$ and $v_k$ is just $a_k$ and $b_k$
      that obtained in Exercise 7.1.
      List $\rho_{k\uparrow}$, $\rho_{-k\downarrow}$ and their conjugates first
      \[
        \begin{array}{l@{~}c@{~}l@{\quad} l@{~}c@{~}l@{\quad}}
          \rho_{k\uparrow}^\dagger & = &
          u_k a_{k\uparrow}^\dagger - v_k a_{-k\downarrow}, &
          \rho_{-k\downarrow}^\dagger & = &
          u_k a_{-k\downarrow}^\dagger + v_k a_{k\uparrow},\\
          \rho_{k\uparrow} & = &
          u_k a_{k\uparrow} - v_k a_{-k\downarrow}^\dagger, &
          \rho_{-k\downarrow} & = &
          u_k a_{-k\downarrow} + v_k a_{k\uparrow}^\dagger.
        \end{array}
      \]
      from which we can obtain the inverse transformations
      \[
        \begin{array}{l@{~}c@{~}l@{\quad} l@{~}c@{~}l@{\quad}}
          a_{k\uparrow}^\dagger & = &
          u_k \rho_{k\uparrow}^\dagger    + v_k \rho_{-k\downarrow}, &
          a_{-k\downarrow}^\dagger & = &
          u_k \rho_{-k\downarrow}^\dagger - v_k \rho_{k\uparrow},\\
          a_{k\uparrow} & = &
          u_k \rho_{k\uparrow}            + v_k \rho_{-k\downarrow}^\dagger, &
          a_{-k\downarrow} & = &
          u_k \rho_{-k\downarrow}         - v_k \rho_{k\uparrow}^\dagger.
        \end{array}
      \]
      Then, using the identities that can be easily obtained
      \[
        u_k^2 - v_k^2 = \frac{t(k)}{E_k}, \quad
        2u_kv_k = \frac{\Delta}{E_k}, \quad
        u_k^2 + v_k^2 = 1, \quad
        E_k = \sqrt{t^2(k) + \Delta^2},
      \]
      and the operator identities of the fermions $a_{k\uparrow}$,
      $a_{-k\uparrow}$ and their conjugates to substitute them into
      $\mathcal H_k$, we have
      \[
        \mathcal H_k
      = E_k(\rho_{k\uparrow}^\dagger  \rho_{k\uparrow} +
            \rho_{-k\uparrow}^\dagger \rho_{-k\uparrow}) +
        (t(k) - E_k) \equiv \mathcal H^*,
      \]
      where the constant term $t(k) - E_k$ is just an energy shift and doesn't
      affect the excitation spectrum. Thus, the operators
      $\rho_{k\uparrow}^\dagger$ and $\rho_{k\downarrow}^\dagger$
      (and their conjugates) create create excitations with energy $E_k$ above
      the BCS ground state.
    \end{proof}
    \item \begin{proof} \let \qed \relax
    Since the fact that Fermionic operators $a_{k\uparrow}$, $a_{-k\downarrow}$
    satisfy the anticommutation relations of, then calculate the same
    anticommutation relations between the operators
    $\rho_{k\uparrow}$, $\rho_{-k\downarrow}$, and their conjugates
    \begin{enumext}
      \item $\{\rho_{k\uparrow}, \rho_{k\uparrow}^\dagger\}$
      \begin{align*}
        \rho_{k\uparrow} \rho_{k\uparrow}^\dagger
      & = u_k^2 a_{k\uparrow} a_{k\uparrow}^\dagger
        - u_k v_k a_{k\uparrow} a_{-k\downarrow}
        - u_k v_k a_{-k\downarrow}^\dagger a_{k\uparrow}^\dagger
        + v_k^2 a_{-k\downarrow}^\dagger a_{-k\downarrow}, \\
        \rho_{k\uparrow}^\dagger \rho_{k\uparrow}
      & = u_k^2 a_{k\uparrow}^\dagger a_{k\uparrow}
        - u_k v_k a_{k\uparrow}^\dagger a_{-k\downarrow}^\dagger
        - u_k v_k a_{-k\downarrow} a_{k\uparrow}
        + v_k^2 a_{-k\downarrow} a_{-k\downarrow}^\dagger.
      \end{align*}
      Adding and using fermionic anticommutators
      \[
        \{a_{k\uparrow}, a_{k\uparrow}^\dagger\} = 1, \quad
        \{a_{-k\downarrow}, a_{-k\downarrow}^\dagger\} = 1,
      \]
      where the cross terms vanish. We get
      \[
        \{\rho_{k\uparrow}, \rho_{k\uparrow}^\dagger\} = u_k^2 + v_k^2 = 1.
      \]
      \item $\{\rho_{-k\downarrow}, \rho_{-k\downarrow}^\dagger\}$
      \begin{align*}
        \rho_{-k\downarrow} \rho_{-k\downarrow}^\dagger
    & = u_k^2 a_{-k\downarrow} a_{-k\downarrow}^\dagger
      + u_k v_k a_{-k\downarrow} a_{k\uparrow}
      + u_k v_k a_{k\uparrow}^\dagger a_{-k\downarrow}^\dagger
      + v_k^2 a_{k\uparrow}^\dagger a_{k\uparrow}, \\
        \rho_{-k\downarrow}^\dagger \rho_{-k\downarrow}
    & = u_k^2 a_{-k\downarrow}^\dagger a_{-k\downarrow}
      + u_k v_k a_{-k\downarrow}^\dagger a_{k\uparrow}^\dagger
      + u_k v_k a_{k\uparrow} a_{-k\downarrow}
      + v_k^2 a_{k\uparrow} a_{k\uparrow}^\dagger.
      \end{align*}
      Adding and using anticommutators
      \[
        \{\rho_{-k\downarrow}, \rho_{-k\downarrow}^\dagger\}
      = u_k^2 + v_k^2 = 1.
      \]
      \item $\{\rho_{k\uparrow}, \rho_{-k\downarrow}^\dagger\}$
      \begin{align*}
        \rho_{k\uparrow} \rho_{-k\downarrow}^\dagger
      & = u_k^2 a_{k\uparrow} a_{-k\downarrow}^\dagger
        + u_k v_k a_{k\uparrow} a_{k\uparrow}
        - u_k v_k a_{-k\downarrow}^\dagger a_{-k\downarrow}^\dagger
        - v_k^2 a_{-k\downarrow}^\dagger a_{k\uparrow}, \\
        \rho_{-k\downarrow}^\dagger \rho_{k\uparrow}
      & = u_k^2 a_{-k\downarrow}^\dagger a_{k\uparrow}
        + u_k v_k a_{k\uparrow} a_{k\uparrow}
        - u_k v_k a_{-k\downarrow}^\dagger a_{-k\downarrow}^\dagger
        - v_k^2 a_{k\uparrow} a_{-k\downarrow}^\dagger.
      \end{align*}
      All terms vanish due to anticommutation
      \[
        \{\rho_{k\uparrow}, \rho_{-k\downarrow}^\dagger\} = 0.
      \]
      \item $\{\rho_{k\uparrow}, \rho_{-k\downarrow}\}$
      \begin{align*}
        \rho_{k\uparrow} \rho_{-k\downarrow}
      & = u_k^2 a_{k\uparrow} a_{-k\downarrow}
        + u_k v_k a_{k\uparrow} a_{k\uparrow}^\dagger
        - u_k v_k a_{-k\downarrow}^\dagger a_{-k\downarrow}
        - v_k^2 a_{-k\downarrow}^\dagger a_{k\uparrow}^\dagger, \\
        \rho_{-k\downarrow} \rho_{k\uparrow}
      & = u_k^2 a_{-k\downarrow} a_{k\uparrow}
        + u_k v_k a_{k\uparrow}^\dagger a_{k\uparrow}
        - u_k v_k a_{-k\downarrow} a_{-k\downarrow}^\dagger
        - v_k^2 a_{k\uparrow}^\dagger a_{-k\downarrow}^\dagger.
      \end{align*}
      Adding and simplifying
      \[
        \{\rho_{k\uparrow}, \rho_{-k\downarrow}\}
      = u_k v_k - u_k v_k = 0.
      \]
      \item For other anticommutators,
      by similar computations or Hermitian conjugation
      \[
        \{\rho_{k\uparrow}^\dagger, \rho_{-k\downarrow}^\dagger\} = 0, \quad
        \{\rho_{k\uparrow}, \rho_{k\uparrow}\} = 0, \qq{etc.}
      \]
      In summary, the operators $\rho_{k\sigma}$,
      $\rho_{k\sigma}^\dagger$, and their conjugates satisfy
      \[
        \{\rho_{k\sigma}, \rho_{k'\sigma'}^\dagger\}
      = \delta_{kk'}\delta_{\sigma\sigma'}, \quad
        \{\rho_{k\sigma}, \rho_{k'\sigma'}\} = 0, \quad
        \{\rho_{k\sigma}^\dagger, \rho_{k'\sigma'}^\dagger\} = 0.
      \]
      Thus, they are purely Fermionic operators. \hfill \ensuremath\square
    \end{enumext}
    \end{proof}
    \item \begin{enumext}
      \item The first term
      \[
          [ \rho_{k'\uparrow}^\dagger \rho_{k'\uparrow},
            \rho_{k\uparrow}^\dagger ]
        = \rho_{k'\uparrow}^\dagger
          \{\rho_{k'\uparrow}, \rho_{k\uparrow}^\dagger\}
        - \{\rho_{k'\uparrow}^\dagger, \rho_{k\uparrow}^\dagger\}
          \rho_{k'\uparrow}
        = \delta_{kk'} \rho_{k'\uparrow}^\dagger \cdot 1 - 0
        = \delta_{kk'} \rho_{k\uparrow}^\dagger.
      \]
      \item The second term
      \[
        [ \rho_{-k'\downarrow}^\dagger \rho_{-k'\downarrow},
          \rho_{k\uparrow}^\dagger ]
      = \rho_{-k'\downarrow}^\dagger
        \{\rho_{-k'\downarrow}, \rho_{k\uparrow}^\dagger\}
      - \{\rho_{-k'\downarrow}^\dagger, \rho_{k\uparrow}^\dagger\}
        \rho_{-k'\downarrow} = 0.
      \]
    \end{enumext}
    Thus, only the term with $k' = k$ contributes
    \[
      [\mathcal H, \rho_{k\uparrow}^\dagger] = E_k \rho_{k\uparrow}^\dagger.
    \]
    This result interrupts that $\rho_{k\uparrow}^\dagger$ is an eigenoperator
    of the Hamiltonian $\mathcal H^*$ with eigenvalue $E_k$. Equivalently,
    $\rho_{k\uparrow}^\dagger$ to an eigenstate of $\mathcal H$ creates an
    excitation of energy $E_k$ measured from the BCS ground state, confirming
    that $\rho_{k\uparrow}^\dagger$ is the quasiparticle creation operator.
    \item The retarded Green's function for the quasiparticle is formulated as
    \[
      \hat G_{k\uparrow}^\text{ret}(E)
    = \ab*<\braket<\rho_{k\uparrow};\rho_{k\uparrow}^\dagger>>_E
    = -\iu \int_{-\infty}^\infty \d t \upe^{\iu Et} \theta(t)
      \braket<\{\rho_{k\uparrow}(t), \rho_{k\uparrow}^\dagger(0)\}>.
    \]
    where we used the ``God-Given'' unit to omit the $\hbar$, and use the
    Heaviside step function to complete the integral range to the whole
    time-axis. Apply the equation of motion for the retarded Green's function
    \[
      E \hat G_{k\uparrow}^\text{ret}(E)
    = \braket<\{\rho_{k\uparrow}, \rho_{k\uparrow}^\dagger\}>
    + \ab*<\braket<[\rho_{k\uparrow}, \mathcal H^*]; \rho_{k\uparrow}^\dagger>>
    = 1 + E_k \ab*<\braket<\rho_{k\uparrow};\rho_{k\uparrow}^\dagger>>_E
    = 1 + E_k \hat G_{k\uparrow}^\text{ret}(E),
    \]
    where $\{\rho_{k\uparrow}, \rho_{k\uparrow}^\dagger\} = 1$.
    Rearrange the equation of motion, we have
    \[
      (E - E_k) \hat G_{k\uparrow}^\text{ret}(E) = \identity, \quad
      \hat G_{k\uparrow}^\text{ret}(E) = \frac1{E - E_k + \iu0^=+}
    \]
    where the $\iu0^+$ enforces retarded boundary conditions.
    \item We have already calculated the retarded Green's function
    \[
      \hat G_{k\uparrow}^\text{ret}(E)
    = \ab*<\braket<\rho_{k\uparrow};\rho_{k\uparrow}^\dagger>>_E^\text{ret}.
    \]
    So, similarly, we can replace $\rho_{k\uparrow}$ and $\rho_{k\uparrow}$
    with $a_{k\uparrow}$, $a_{-k\downarrow}$ and their conjugates.
    By substituting the inverse transformations obtained in (a)
    \[
      \begin{array}{l@{~}c@{~}l@{\quad} l@{~}c@{~}l@{\quad}}
        a_{k\uparrow}^\dagger & = &
        u_k \rho_{k\uparrow}^\dagger    + v_k \rho_{-k\downarrow}, &
        a_{-k\downarrow}^\dagger & = &
        u_k \rho_{-k\downarrow}^\dagger - v_k \rho_{k\uparrow},\\
        a_{k\uparrow} & = &
        u_k \rho_{k\uparrow}            + v_k \rho_{-k\downarrow}^\dagger, &
        a_{-k\downarrow} & = &
        u_k \rho_{-k\downarrow}         - v_k \rho_{k\uparrow}^\dagger.
      \end{array}
    \]
    we can obtain the following Green functions
    \[
      \begin{array}{l@{~}c@{~}l@{~}c@{~}l@{\quad}l}
        G_{\uparrow\uparrow}^\text{ret}(k, E)
      & = & \ab*<\braket<a_{k\uparrow}; a_{k\uparrow}^\dagger>>
      & = & \frac{u_k^2}{E - E_k + \iu0^+} + \frac{v_k^2}{E + E_k + \iu0^+},
      & \text{Normal Green's function},\\
        G_{\downarrow\downarrow}^\text{ret}(k, E)
      & = & \ab*<\braket<a_{k\downarrow}; a_{k\downarrow}^\dagger>>
      & = & \frac{u_k^2}{E - E_k + \iu0^+} + \frac{v_k^2}{E + E_k + \iu0^+},
      & \text{Normal Green's function},\\
        G_{\uparrow\downarrow}^\text{ret}(k, E)
      & = & \ab*<\braket<a_{k\uparrow}; a_{-k\downarrow}>>
      & = & -\frac\Delta{E^2 - E_k^2 + \iu0^+},
      & \text{Anomalous Green's function},\\
        \tilde G_{\uparrow\downarrow}^\text{ret}(k, E)
      & = & \ab*<\braket<a_{k\uparrow}^\dagger; a_{-k\downarrow}^\dagger>>
      & = & \frac\Delta{E^2 - E_k^2 + \iu0^+},
      & \text{Anomalous Green's function},
      \end{array}
    \]
    and other combinations vanish due to symmetry.
  \end{enumext}
\end{solution}