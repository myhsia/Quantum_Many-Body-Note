% !TeX root = ../main.tex

\section{Homework \#4 [2025-09-23]}

\begin{problem}[Exercise 5.1 of Coleman]
  A particle with $S = \frac12$ is placed in a large magnetic field
  $\bm B = (B_1 \cos \omega t, B_1 \sin \omega_t, B_0)$,
  where $B_0 \gg B_1$.
  \begin{enumext}[label = (\alph*)]
    \item Treating the oscillating part of the Hamiltonian as the interaction,
    write down the Schr\"odinger equation in the interaction representation.
    \item Find $\hat U(t) = \mathcal T
    \exp[-\iu \int_{-\infty}^t \mathcal H_\text{int}(t') \d t']$
    by whatever method proves most convenient.
    \item If the particle starts at time $t = 0$ in the state
    $S_z = -\frac12$, what is the probability it is in this state at time $t$?
  \end{enumext}
\end{problem}
\begin{solution}\leavevmode
  \begin{enumext}[label = (\alph*)]
    \item The Hamiltonian is  
    \[
      \hat H(t) = -\gamma \bm S \cdot \mathbf{B}(t)
    = -\gamma B_0 S_z - \gamma B_1\ab(S_x \cos \omega t + S_y \sin \omega t),
    \]  
    with $\bm S = \frac\hbar2 \bm \sigma$.
    Denote $\hat H_0 = -\gamma B_0 S_z$ and 
    $\hat H_1(t) = -\gamma B_1\ab(S_x \cos \omega t + S_y \sin \omega t)$,
    we obtain
    \[
      \hat H_1(t) = -\frac{\gamma B_1}{2}
      \ab(S_+ \upe^{-\iu\omega t} + S_- \upe^{\iu\omega t}).
    \]
    where $S_\pm = S_x \pm i S_y$.
    In the interaction picture, the interaction Hamiltonian is expressed as
    \[
      \hat H_{1,I}(t) = \upe^{\iu\hat H_0 t/\hbar}
      \hat H_1(t) \upe^{-\iu\hat H_0 t/\hbar},
    \]
    with the BCH theory, using
    $e^{\iu\hat H_0 t/\hbar} S_\pm \upe^{-\iu\hat H_0 t/\hbar}
    = S_\pm \upe^{\mp \iu\omega_0 t}$, where $\omega_0 = \gamma B_0$,
    we obtain  
    \[
      \hat H_{1,I}(t) = -\frac{\gamma B_1}{2}
      \ab(S_+ \upe^{-\iu(\omega + \omega_0) t}
        + S_- \upe^{\iu(\omega + \omega_0) t}).
    \]
    So, the Schr\"odinger equation in the interaction representation is
    \[
      \iu \hbar \odv{\ket|\psi_I(t)>}t = -\frac{\gamma B_1}{2}
      \ab(S_+ \upe^{-\iu(\omega + \omega_0) t}
        + S_- \upe^{\iu(\omega + \omega_0) t}) \ket|\psi_I(t)>,
    \]
    \item Let $\Delta = \omega + \omega_0$,
    $\Omega = -\frac{\gamma B_1}{2}$. Then
    \[
      \hat H_{1,I}(t)
    = \Omega\ab(S_+ \upe^{-\iu\Delta t} + S_- \upe^{\iu\Delta t})
    = 2\Omega \upe^{-\iu\Delta t S_z} S_x \upe^{\iu\Delta t S_z}.
    \]
    The time-evolution operator is  
    \[
      \hat U(t) = \mathcal T
      \exp\ab[ -\frac\iu\hbar \int_{-\infty}^t \hat H_{1,I}(t') \d t'].
    \]
    Using the identity for time-ordered exponentials of this form, we find  
    \[
      \hat U(t) = \upe^{-\iu\Delta t S_z} \exp\ab[
                  -\frac\iu\hbar \ab(2\Omega S_x - \Delta S_z) t]
                = \upe^{-\iu(\omega + \omega_0) t S_z}
                  \exp\ab[\frac\iu\hbar t
                      \ab(\gamma B_1 S_x + (\omega + \omega_0) S_z)].
    \]
    \item Assume in the interaction picture,
    $\ket|\psi_I(0)> = \ket|\downarrow>$.
    The amplitude to remain in $\ket|\downarrow>$ is
    \[
      A(t) = \braket<\downarrow|\hat U(t)|\downarrow>.
    \]
    Let $\Omega_R = \gamma B_1$, $\tilde{\Delta} = \omega + \omega_0$, and define the effective Hamiltonian according to (b)
    \[
      H_{\text{eff}} = -\Omega_R S_x - \tilde{\Delta} S_z,
    \]
    then  
    \[
      \hat U(t) = \upe^{-i\tilde{\Delta} t S_z} \upe^{-\iu H_{\text{eff}} t/\hbar}.
    \]
    Since $\ket|\downarrow>$ is an eigenstate of $S_z$ with eigenvalue $-\hbar/2$,  
    \[
    e^{-\iu\tilde{\Delta} t S_z} \ket|\downarrow> = \upe^{\iu\tilde{\Delta} t/2} \ket|\downarrow>,
    \]  
    so  
    \[
    A(t) = \upe^{\iu\tilde{\Delta} t/2}
          \braket<\downarrow|\upe^{-\iu H_\text{eff} t/\hbar}|\downarrow>, \qq{and}
      H_\text{eff} = -\frac12 \ab(\Omega_R \sigma_x + \tilde{\Delta} \sigma_z).
    \]
    The eigenvalues of $H_{\text{eff}}$ are $\pm \frac{\lambda}{2}$, where $\lambda = \sqrt{\Omega_R^2 + \tilde{\Delta}^2}$. So
    \[
      A(t) = \upe^{\iu\tilde{\Delta} t/2}
              \ab[\cos\ab(\frac{\lambda t}{2})
            - \iu\frac{\tilde{\Delta}}{\lambda} \sin\ab(\frac{\lambda t}{2})],
    \]
    and we obtain the survival probability is  
    \[
      P(t) = |A(t)|^2 
    = 1 - \frac{(\gamma B_1)^2}{(\gamma B_1)^2 + (\omega + \gamma B_0)^2}
      \sin^2\ab(\frac12 \sqrt{(\gamma B_1)^2 + (\omega + \gamma B_0)^2} t).
    \]
  \end{enumext}
\end{solution}

\newpage

\begin{problem}[Exercise 3.1.1 of Nolting]
  For the non-interacting electron gas ($H_e$) and
  for the non-interacting phonon gas ($H_p$),
  \[
    H_e = \sum_{k,\sigma} \epsilon(k) a_{k\sigma}^\dagger a_{k\sigma}; \qq{and}
    H_p = \sum_{q,r} \hbar \omega_r(q) \ab(b_{qr}^\dagger b_{qr} + \frac12),
  \]
  compute the time dependence of the annihilation operators
  $a_{k\sigma}(t)$, $b_{qr}(t)$ in the Heisenberg picture.
\end{problem}
\begin{solution}
  We just need to apply EOM to $a_{k\sigma}$ and $b_{qr}$.
  \begin{enumext}[label = (\alph*)]
    \item Annihilation operator $a_{k\sigma}(t)$\\ 
    Calculate the commutator $[H_e, a_{k\sigma}]$ first
    \begin{align*}
      [H_e, a_{k\sigma}] & = \sum_{k',\sigma'}
      \epsilon(k')[a_{k'\sigma'}^\dagger a_{k'\sigma'}, a_{k\sigma}]\\
      & = \sum_{k',\sigma'}
      \epsilon(k') \ab(a_{k'\sigma'}^\dagger \cancel{[a_{k'\sigma'}, a_{k\sigma}]}
      + [a_{k'\sigma'}^\dagger, a_{k\sigma}] a_{k'\sigma'})
      = -\epsilon(k) a_{k\sigma}.
    \end{align*}
    Then the EOM becomes
    \[
      \odv{a_{k\sigma}(t)}t = \frac\iu\hbar [-\epsilon(k) a_{k\sigma}(t)].
    \]
    It's trivial that
    \[
      a_{k\sigma}(t) = a_{k\sigma}(0) \upe^{-\iu\epsilon(k)t/\hbar}.
    \]
    \item Similarly, the commutator
    \[
      [H_p, b_{qr}] = -\hbar \omega_r(q) b_{qr},
    \]
    and the EOM becomes
    \[
      \odv{b_{qr}(t)}t = -\iu\omega_r(q) b_{qr}(t),
    \]
    and it's trivial that
    \[
      b_{qr}(t) = b_{qr}(0) \upe^{-\iu\omega_r(q)t}.
    \]
  \end{enumext}
\end{solution}

\begin{problem}
  Consider an arbitrary \textbf{time-dependent} operator in Schr\"odinger picture
  $\hat A_S(t)$ then re-express $\hat A_S(t)$ in Heisenberg picture
  as $\hat A_H(t)$ and in Dirac (interaction) picture as $\hat A_D(t)$.
  \begin{enumerate}
    \item Show that $\braket<\psi_H|A_H(t)|\psi_H> \overset!=
    \braket<\psi_S(t)|A_S|\psi_S(t)>$;
    \item Derive the Heisenberg equation-of-motion for $\hat A_H(t)$
    which should read;
    \item Similarly, derive the equation-of-motion in Dirac (interaction)
    picture for $\hat A_D(t)$.
  \end{enumerate}
\end{problem}
\begin{solution}\leavevmode
  \begin{enumext}[label = (\alph*)]
    \item In Heisenberg picture, $\psi_H = \psi_S(0)$, so we have
    \[
      \braket<\psi_H|\hat A_H(t)|\psi_H>
    = \braket<\psi_S(0)|\hat U^\dagger(t,0) \hat A_S(t) \hat U(t,0)|\psi_S(0)>.
    \]
    Since $\ket|\psi_S(t)> = \hat U(t,0) \ket|\psi_S(0)>$, so
    \[
      \braket<\psi_S(t)|\hat A_S(t)|\psi_S(t)>
      = \braket<\psi_S(0)|\hat U^\dagger(t,0) \hat A_S(t) \hat U(t,0)|\psi_S(0)>,
    \]
    which is exactly $\braket<\psi_H|\hat A_H(t)|\psi_H>$. So
    \[
      \braket<\psi_H|A_H(t)|\psi_H> = \braket<\psi_S(t)|A_S|\psi_S(t)>.
    \]
    \item Since the operator in the Heisenberg picture can be expressed as
    \[
      \hat A_H = \hat U^\dagger(t,0) \hat A_S(t) \hat U(t,0),
    \]
    and its differentiate
    \[
      \odv*{\hat A_H(t)}t = \pdv{\hat U^\dagger}t \hat A_S(t) U
    + \hat U^\dagger \pdv{\hat A_S(t)}t \hat U
    + \hat U^\dagger \hat A_S(t) \pdv{\hat U}t.
    \]
    Substitute the equation that $\hat U$ satisfies under the Heisenberg picture
    \[
      \iu\hbar \pdv*{\hat U}t = \hat H_S \hat U,
    \]
    we have
    \begin{align*}
      \odv*{\hat A_H}t & = \ab(-\frac1{\iu\hbar} \hat U^\dagger H_S) A_SU
    + \hat U^\dagger \pdv{\hat A_S(t)}t \hat U
    + \hat U^\dagger \hat A_S \ab(\frac1{\iu\hbar} H_S \hat U)\\
    & = \frac1{\iu\hbar} [-\hat U^\dagger H_S \hat A_S \hat U +
      \hat U^\dagger \hat A_S H_S \hat U] + \hat U^\dagger \dot A_S \hat U,
    \end{align*}
    where $-\hat U^\dagger H_S \hat A_S \hat U = -H_H\hat A_H$,
    $\hat U^\dagger \hat A_S H_S \hat U = \hat A_HH_H$,
    and $\hat U^\dagger \dot A_S \hat U = \ab(\pdv{A_S}t)_H$.
    So we obtain the Heisenberg equation of motion
    \[
      \odv*{A_H(t)}t = \frac1{\iu\hbar} [\hat A_H(t), H_H(t)] + \ab(\pdv{A_S}t)_H.
    \]
    \item Similarly from
    \[
      A_I(t) = \hat U_0^\dagger(t,0) \hat A_S(t) \hat U_0(t,0),
    \]
    we have the differentiate
    \[
      \odv{\hat A_I}t = \pdv{\hat U_0^\dagger}t A_S \hat U_0
    + \hat U_0^\dagger \pdv{\hat A_S}t \hat U_0 + \hat U_0^\dagger \hat A_S \pdv{\hat U_0}t.
    \]
    Substitute the equation that $\hat U$ satisfies
    \[
      \iu\hbar \pdv{\hat U_0}t = H_{0,S} \hat U_0,
    \]
    we have
    \[
      \odv{\hat A_I}t = -\frac1{\iu\hbar}
      \ab[\hat U_0^\dagger H_{0,s} \hat A_s \hat U_0
      + \hat U_0^\dagger \dot{A_S} \hat U_0
      + \frac1{\iu\hbar} \hat U_0^\dagger A_S H_{0,s} \hat U_0].
    \]
    Since $H_{0,I}(t) = \hat U_0^\dagger H_{0,S} \hat U_0$
    and $\hat A_I = \hat U_0^\dagger \hat A_S \hat I_0$,
    we obtain the Dirac picture equation of motion
    \[
      \odv*{A_I(t)}t = \frac1{\iu\hbar} [A_I(t), H_{0,I}(t)]
    + \ab(\pdv{\hat A_S}t)_I.
    \]
  \end{enumext}
\end{solution}